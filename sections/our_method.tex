\section{Used method}
\label{sec:used_method}

The time dependent Schrödinger equation as written in \ref{eq:schrodinger}, is a linear partial differential equation.
The $H$ operator and a given $\psi(\vec{r}; t_0)$ initial state fully determines the $\psi(
vec{r}; t)$ wave function.
Formal solution of equation~\ref{eq:schrodinger} can be given in the form of equation~\ref{eq:schrodinger_formal_solution}.
\begin{equation}
	\label{eq:schrodinger_formal_solution}
	\psi(\vec{r}; t) = U\psi(\vec{r}, t_0)\;\;\;\;\;\;\;\; U = e^{-iH(t-t_0)}
\end{equation}
Here $U$ is the time development operator.
The exponent containing $H$ can not be trivially factored.
We can use an approximation described in equation~\ref{eq:solution_approximation}.
\begin{equation}
	\label{eq:solution_approximation}
	e^{-i(K + V)\delta{}t} \approx e^{-iK\delta{}t/2}e^{-iV\delta{}t}e^{-iK\delta{}t/2}
\end{equation}
The error of this approximation is $O\left[(\delta{}t)^3 \right]$.
We have to consider this when designing the time resolution of the simulation.
The evolution operator is now split into three consecutive steps.
These are a free propagation for time $\delta{}t / 2$, a potential only propagation for $\delta{}t$ and another free propagation for $\delta{}t / 2$.
Since the effect of $V$ is a multiplication with $V(\vec{r})$ we can write the potential propagator as a multiplication with $e^{-iV(\vec{r})\delta{}t}$.
The effect of the free propagator is simple when applied in the $k$ momentum space.
In this case it is simply a multiplication with $e^{i|k|^2\delta{}t/4}$.
To use this formula we have to convert from real space to momentum space and backward.
To do this we must calculate the Fourier transform of the wave function.
On the computer this is approximated by calculating the Discrete Fourier Transform (DFT).
DFT can be implemented with an algorithmic complexity of $O\left[n\,log(n)\right]$  with the Fast Fourier Transform (FFT) algorithm.
This method utilises the symmetries naturally occurring while calculating the DFT by definition.
Namely that you can reuse some values by pairing the odd and even indexed members of the input vector.

Given a $\psi_n$ current state, the next state occurring after $\delta{}t$ time can be obtained by executing the following steps.
\begin{enumerate}
	\item{$\psi_n^{(1)} := FFT^{-1}\left[ P_{kinetic}(\delta{}t/2)\;FFT(\psi_n) \right]$}
	\item{$\psi_n^{(2)} := P_{potential}(\delta{}t)\;\psi_n^{(1)}$}
	\item{$\psi_{n+1} := FFT^{-1}\left[ P_{kinetic}(\delta{}t/2)\;FFT(\psi_n^{(2)}) \right]$}
\end{enumerate}
Here $P_{kinetic}(\delta{}t/2)$ is the previously discussed free propagator applied by multiplying with $e^{i|k|^2\delta{}t/4}$, and $P_{potential}(\delta{}t)$ is the potential propagator applied by multiplying with $e^{-iV(\vec{r})\delta{}t}$.

To obtain a convergent and aliasing free simulation one must choose small enough $\delta{}x,\; \delta{}y$ grid resolution that is smaller than the \textit{de Broglie} wavelength of the simulated system.

The output of the simulation is the probability density function of the simulated volume.
This can be obtained by calculating the square of the absolute value of the wave function for each position as seen in equation~\ref{eq:probability_density}.
\begin{equation}
	\label{eq:probability_density}
	p(\vec{r}; t) = |\psi(\vec{r}, t)|^2
\end{equation}


