\section{Discussion}

Seeing all the images generated by our implementation, it is safe to say that the described method has a lot of potential when it comes to simulating and visualizing quantum wave functions.

To measure the stability of the simulation we integrated the probability density over the simulated volume to check the deviation from the awaited
$P = 1$.
This test resulted in fairly low additive error of magnitude $\approx 10^{-13}$ even after $1000$ time steps.

In order to maintain the convergence of the numeric approximation described in section~\ref{sec:used_method} we have to choose small enough time steps.
To battle the $O\left[ (\delta{}t)^3 \right]$ error of the approximated time development operator we used the formula~\ref{eq:max_delta_time} published in \cite{mark2020webschrodinger} multiplied by a heuristically chosen $0.1$.
\begin{equation}
	\label{eq:max_delta_time}
	\delta{}t < \frac{4}{\pi} \frac{(\delta{}x,\,\delta{}y)^2}{D}
\end{equation}
Here $D$ stand for the number of dimensions simulated. In our case this is equal to $3$.

The time required to run the simulation for all the $1000$ frames took multiple hours.
Partially this is due to the slow plotting of the used isosurface visualisation.
The bottleneck of core algorithm are the multiple FFT runs.

In the future we want to further develop the method.
We plan on experimenting with advanced wave function setups.
We are going to implement a custom ray tracer to create nicer visualizations of the simulation results.

We will explore the possibilities of using this method to simulate atomic structures and more advanced molecules.


